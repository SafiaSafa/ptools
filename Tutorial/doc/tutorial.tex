\documentclass[12pt,a4paper]{article}
\usepackage{ae}
\usepackage{aecompl}
%\usepackage[cm]{aeguill}
\textwidth 16.5cm \textheight 26cm \oddsidemargin -0.5cm \topmargin -2cm
\usepackage{graphicx} % figures

\begin{document}


\title{PTools tutorial}
\maketitle

\thispagestyle{empty}

\begin{figure}[htbp]
\center
\includegraphics*[width=0.80\linewidth]{img/docking.png}
\end{figure}


\vspace*{3cm}

\noindent
This tutorial presents the PTools library features and its docking application ATTRACT.

\newpage

\tableofcontents{}

\newpage

\section{Set up, compilation and installation}

\subsection{Set up: Dell D420 laptop, Debian Lenny, 32-bit}
This example installation has been performed on a Dell D420 laptop (Intel
Core Duo 1.2~GHz, Debian Lenny, linux kernel 2.6.22-2-686).

The basic requirements are:
\begin{itemize}
\item Python 2.4 or 2.5 and its development library (python2.4-dev or python2.5-dev)
\item gcc (4.x)
\item gfortran (4.x)
\end{itemize}

\paragraph{SCons (make substitute):} {\tt apt-get install scons} \\
(1.34.1-11 version installed)


\paragraph{Boost C++ Libraries} {\tt apt-get install libboost-dev} \\
(1.34.1-11 version installed)


\paragraph{Boost Python Library:} {\tt apt-get install libboost-python} \\
(1.34.1-11 version installed)


\paragraph{libboost Python dev:} {\tt apt-get install libboost-python-dev} \\
(1.34.1-11 version installed)

\paragraph{Subversion:} {\tt apt-get install subversion subversion-tools} \\
(1.4.6dfsg1-2 version installed)

\paragraph{gccxml:} {\tt apt-get install gccxml} \\
(0.9.0+cvs20071228-2 version installed)

Some Ubuntu linux distribution have an incompatible gccxml 0.7 version that crashes when parsing headers files. The solution is to install gccxml from the official CVS repository ( {\tt http://www.gccxml.org/HTML/Download.html} )


\paragraph{Pyplusplus and Pygccxml: }
The homepage of {\tt pyplusplus} and {\tt pygccxml} projects is 
{\tt http://www.language-binding.net/pyplusplus/pyplusplus.html}. From the
download section \footnote{\tt https://sourceforge.net/project/showfiles.php?group\_id=118209}, 
get the files {\tt pygccxml-0.9.5.zip} and {\tt Py++-0.9.5.zip}.

\begin{verbatim}
unzip pygccxml-0.9.5.zip
cd pygccxml-0.9.5/
python setup.py build
python setup.py install --prefix=$HOME/soft
\end{verbatim}

\begin{verbatim}
unzip Py++-0.9.5.zip 
cd Py++-0.9.5/
python setup.py build
python setup.py install --prefix=$HOME/soft
\end{verbatim}

In your {\tt \$HOME/.bashrc} file, then add:
\begin{verbatim}
export PATH=PATH:$HOME/soft/bin/
export PYTHONPATH=$HOME/soft/lib/python2.4/site-packages/
\end{verbatim}

In a Python shell (obtained with the {\tt python} command), test the installation
of PyPlusPlus:
\begin{verbatim}
>>> import pyplusplus
\end{verbatim}


\subsection{Set up: Dell Precision T7400, Linux Fedora Core 9, 64-bit}
This example installation has been performed on a Dell Precision T7400 (Intel
Xeon quad-core 2.3~GHz, Linux Fedora Core 9, linux kernel 2.6.25.6-55.FC9.x86\_64).

The basic requirements are:
\begin{itemize}
\item Python 2.5 and its development library python2.5-dev
\item gcc (4.x)
\item gfortran (4.x)
\end{itemize}

\paragraph{SCons (make substitute):}
The scons package provided by Fedora is not up-to-date enough to link Fortran and C++ together and then rise an
error. The last version of scons is obtained at the homepage of the project ({\tt  http://www.scons.org/}).
From the download section \footnote{\tt http://sourceforge.net/project/showfiles.php?group\_id=30337}, 
get the stable file {\tt scons-0.98.5-1.noarch.rpm}.

\begin{verbatim}
 rpm -ivh scons-0.98.5-1.noarch.rpm
\end{verbatim}

\paragraph{Boost C++ Libraries:} {\tt yum install boost boost-devel} \\
(1.34.1-13 version installed)


\paragraph{Python dev:} {\tt yum install python-devel} \\
(2.5.1-26 version installed)

\paragraph{Subversion:} {\tt yum install subversion} \\
(1.4.6-7 version installed)

\paragraph{gccxml:}The homepage of {\tt gccxml} project is {\tt http://www.gccxml.org}. 
To install gccxml from the official CVS repository \footnote{\tt http://www.gccxml.org/HTML/Download.html}.

\begin{verbatim}
cvs -d :pserver:anoncvs@www.gccxml.org:/cvsroot/GCC_XML login
\end{verbatim}
(just press enter when prompted for a password)\\
Follow this command by checking out the source code:
\begin{verbatim}
cvs -d :pserver:anoncvs@www.gccxml.org:/cvsroot/GCC_XML co gccxml
mkdir gccxml-build
cd gccxml-build
cmake ../gccxml -DCMAKE_INSTALL_PREFIX:PATH=/installation/path
make
make install
\end{verbatim}
The "-DC\_MAKE\_INSTALL\_PREFIX" option can be left off if you want to use "/usr/local" as the installation prefix.


\paragraph{Pyplusplus and Pygccxml: }
The homepage of {\tt pyplusplus} and {\tt pygccxml} projects is 
{\tt http://www.language-binding.net/pyplusplus/pyplusplus.html}. From the
download section \footnote{\tt https://sourceforge.net/project/showfiles.php?group\_id=118209}, 
get the files {\tt pygccxml-0.9.5.zip} and {\tt Py++-0.9.5.zip}.

\begin{verbatim}
unzip pygccxml-0.9.5.zip
cd pygccxml-0.9.5/
python setup.py build
python setup.py install --prefix=$HOME/soft
\end{verbatim}

\begin{verbatim}
unzip Py++-0.9.5.zip 
cd Py++-0.9.5/
python setup.py build
python setup.py install --prefix=$HOME/soft
\end{verbatim}

In your {\tt \$HOME/.bashrc} file, then add:
\begin{verbatim}
export PATH=PATH:$HOME/soft/bin/:/usr/lib/gcc/x86_64-redhat-linux/3.4.6/
export PYTHONPATH=$HOME/soft/lib/python2.4/site-packages/
\end{verbatim}

In a Python shell (obtained with the {\tt python} command), test the installation
of PyPlusPlus:
\begin{verbatim}
>>> import pyplusplus
\end{verbatim}

\subsection{Set up: Dell Precision 690, Linux Fedora Core 7, 64-bit}
This example installation has been performed on a Dell Precision 690 (Intel
Xeon bi-quad core 1.9~GHz, Linux Fedora Core 7, linux kernel 2.6.23.17-88.fc7).

The basic requirements are:
\begin{itemize}
\item Python 2.5 and its development library python2.5-dev
\item gcc (4.x)
\item gfortran (4.x)
\end{itemize}

\paragraph{SCons (make substitute):}
The scons package provided by Fedora is not up-to-date enough to link Fortran and C++ together and then rise an
error. The last version of scons is obtained at the homepage of the project ({\tt  http://www.scons.org/}).
From the download section \footnote{\tt http://sourceforge.net/project/showfiles.php?group\_id=30337}, 
get the stable file {\tt scons-0.98.5-1.noarch.rpm}.

\begin{verbatim}
 rpm -ivh scons-0.98.5-1.noarch.rpm
\end{verbatim}

\paragraph{Boost C++ Libraries:} {\tt yum install boost boost-devel} \\
(1.33.1-15.fc7 version installed)


\paragraph{Python dev:} {\tt yum install python-devel} \\
(2.5-15.fc7 version installed)

\paragraph{Subversion:} {\tt yum install subversion} \\
(1.4.4-1.fc7 version installed)

\paragraph{gccxml:}The homepage of {\tt gccxml} project is {\tt http://www.gccxml.org}. 
To install gccxml from the official CVS repository \footnote{\tt http://www.gccxml.org/HTML/Download.html}.

\begin{verbatim}
yum install cvs
\end{verbatim}
1.11.22-9.1.fc7 version installed

\begin{verbatim}
yum install cmake
\end{verbatim}
2.4.8-1.fc7 version installed

\begin{verbatim}
cvs -d :pserver:anoncvs@www.gccxml.org:/cvsroot/GCC_XML login
\end{verbatim}
(just press enter when prompted for a password)\\
Follow this command by checking out the source code:
\begin{verbatim}
cvs -d :pserver:anoncvs@www.gccxml.org:/cvsroot/GCC_XML co gccxml
mkdir gccxml-build
cd gccxml-build
cmake ../gccxml -DCMAKE_INSTALL_PREFIX:PATH=/installation/path
make
make install
\end{verbatim}
The "-DC\_MAKE\_INSTALL\_PREFIX" option can be left off if you want to use "/usr/local" as the installation prefix.


\paragraph{Pyplusplus and Pygccxml: }
The homepage of {\tt pyplusplus} and {\tt pygccxml} projects is 
{\tt http://www.language-binding.net/pyplusplus/pyplusplus.html}. From the
download section \footnote{\tt https://sourceforge.net/project/showfiles.php?group\_id=118209}, 
get the files {\tt pygccxml-0.9.5.zip} and {\tt Py++-0.9.5.zip}.

\begin{verbatim}
unzip pygccxml-0.9.5.zip
cd pygccxml-0.9.5/
python setup.py build
python setup.py install --prefix=$HOME/soft
\end{verbatim}

\begin{verbatim}
unzip Py++-0.9.5.zip 
cd Py++-0.9.5/
python setup.py build
python setup.py install --prefix=$HOME/soft
\end{verbatim}

In your {\tt \$HOME/.bashrc} file, then add:
\begin{verbatim}
export PATH=PATH:$HOME/soft/bin/:/usr/lib/gcc/x86_64-redhat-linux/3.4.6/
export PYTHONPATH=$HOME/soft/lib/python2.4/site-packages/
\end{verbatim}

In a Python shell (obtained with the {\tt python} command), test the installation
of PyPlusPlus:
\begin{verbatim}
>>> import pyplusplus
\end{verbatim}

\paragraph{g2c library: }

\begin{verbatim}
yum install g2clib-devel
\end{verbatim}
1.0.5-3.fc7 version installed


\subsection{Source download with subversion}

From the local {\tt \$HOME/soft/} directory, download the full PTools sources with subversion (use the {\it checkout} option):
\begin{verbatim}
svn co http://svn-lbt.ibpc.fr/svn/PTools/ptools ptools
\end{verbatim}

\begin{itemize}
\item for the first use, press {\it Enter} and then indicate your login and password
\item both login and password are stored (without encryption) in the {\tt \$HOME/.subversion} directory
\end{itemize}

PTools source updates are then obtained by:
\begin{verbatim}
svn update
\end{verbatim}
from the {\tt ptools/} or {\tt ptools/trunk} directories.


\subsection{Compilation}
From the main directory ({\tt trunk}) of the PTools project, create the Python/C++ interface:
\begin{verbatim}
python interface.py
\end{verbatim}

Compile then the library:
\begin{verbatim}
scons
\end{verbatim}

Note that {\tt scons -j2} compiles with two processors in parallel.

If scons complains about the g2c library, you may do (as root):
\begin{verbatim}
ln -s /usr/lib64/libg2c.so.0 /usr/lib64/libg2c.so
\end{verbatim}

\subsection{Final test and further documentation}

In the {\tt Test} directory, one can test the compilation worked:
\begin{verbatim}
python unittest1.py
\end{verbatim}

The expected output is:
\begin{verbatim}
....... 
---------------------------------------------------------------------- 
Ran 7 tests in 0.813s 

OK 
\end{verbatim}

Further document may be obtained from the Trac server {\tt http://svn-lbt.ibpc.fr/PTools/wiki}. 
The server access is controled by the same login/password of the subversion server. The {\tt Timeline}
and {\tt Browse Source} sections are usually very usefull.

A {\tt README} file is also available on line 
\footnote{\tt http://svn-lbt.ibpc.fr/PTools/browser/ptools/trunk/Tutorial/README} 
or locally \footnote{\tt \$HOME/soft/ptools/trunk/Tutorial/README}.


\section{PTools library usages and capabilities}

\subsection{Directly from C++}

\subsection{From Python through the C++ binding}

\vspace*{1cm}
{\bf Explain here the different level of objects between rigidbody, selection, atom and points}
\vspace*{1cm}

If PTools has been installed in the {\tt \$HOME/soft/ptools/branches/1.0rc/} directory, 
declare it in your PYTHONPATH :

\begin{verbatim}
export PYTHONPATH=$PYTHONPATH:$HOME/soft/ptools/branches/1.0rc/
\end{verbatim}
or add this line at the end of your {\tt \$HOME/.bashrc} file.


From the Python interpreter or in a Python script, first load the PTools library:
\begin{verbatim}
from ptools import *
\end{verbatim}


\subsubsection{Rigidbody objects}

\paragraph{Load PDB file into a rigidbody object.}
\begin{verbatim}
pdb = Rigidbody("1BTA.pdb")
\end{verbatim}


\paragraph{Number of atoms}
\begin{verbatim}
pdb.Size()
\end{verbatim}


\paragraph{Maximum distance from geometric center} in \AA.
\begin{verbatim}
pdb.Radius()
\end{verbatim}


\paragraph{Radius of gyration} in \AA.
\begin{verbatim}
pdb.RadiusGyration()
\end{verbatim}


\paragraph{Structure translation.}
First create a translation vector as a Coord3D object (for instance 5, 0, 1):
\begin{verbatim}
trans = Coord3D(5, 0, 1)
\end{verbatim}
Then, apply the translation vector:
\begin{verbatim}
pdb.Translate(trans)
\end{verbatim}


\paragraph{Center structure to origin.}
\begin{verbatim}
pdb.CenterToOrigin()
\end{verbatim}


\paragraph{Save structure as PDB file.}
\begin{verbatim}
WritePDB(pdb, "1BTA_centered.pdb")
\end{verbatim}


\subsubsection{Selection objects}

\paragraph{Selection of CA atoms.}
\begin{verbatim}
sel_ca = pdb.CA()
\end{verbatim}


\paragraph{Selection of backbone atoms.}
\begin{verbatim}
sel_bkbn = pdb.Backbone()
\end{verbatim}


\paragraph{Selection by chain.}
\begin{verbatim}
sel_chainA = pdb.SelectChainId("A")
sel_chainB = pdb.SelectChainId("B")
\end{verbatim}


\paragraph{Selection of a range of residues.}
\begin{verbatim}
sel_res = pdb.SelectResRange(10, 20)
\end{verbatim}


\paragraph{Selection number of atoms.}
\begin{verbatim}
sel_res.Size()
\end{verbatim}



\paragraph{Selection reunion.}
\begin{verbatim}
sel_chainAB = sel_chainA | sel_chainB
\end{verbatim}
or directly
\begin{verbatim}
sel_chainAB = pdb.SelectChainId("A") | pdb.SelectChainId("B")
\end{verbatim}


\paragraph{Selection to rigidbody conversion.}
\begin{verbatim}
ca_trace = sel_ca.CreateRigid()
\end{verbatim}



\section{Docking with PTools: ATTRACT}

This part is illustrated by the docking of 1F34 complex.

\subsection{Protein--protein complex: 1F34}

The 1F34 complex is made of two partners. Chain A: 326 residues, 2433 atoms and chain B: 148 residues, 1074 atoms.

\subsubsection{Extraction of the docking partners}

Before docking, one has to separate both partners. This is possible with visualisation
software such as Pymol \footnote{\tt http://pymol.sourceforge.net/} or VMD \footnote{\tt http://www.ks.uiuc.edu/Research/vmd/}, 
and also directly with the PTools library.

Within the Python interpreter, first load the PTools library:
\begin{verbatim}
from ptools import *
\end{verbatim}

Read the PDB file 1F34.pdb:
\begin{verbatim}
pdb=Rigidbody("1F34.pdb")
\end{verbatim}

The chain selection allows the separation between chain A and B.
\begin{verbatim}
selectA=pdb.SelectChainId("A")
selectB=pdb.SelectChainId("B")
\end{verbatim}

Creation of both chains as independant rigid body and file saving as a PDB:
\begin{verbatim}
protA=selectA.CreateRigid()
protB=selectB.CreateRigid()
WritePDB(protA,"1F34_A.pdb")
WritePDB(protB,"1F34_B.pdb")
\end{verbatim}
Or more quickly:
\begin{verbatim}
WritePDB(selectA.CreateRigid(),"1F34_A.pdb")
WritePDB(selectB.CreateRigid(),"1F34_B.pdb")
\end{verbatim}

\subsubsection{Coarse grain reduction}

This step translates all atom molecule into coarse grain (reduced) molecule for a further docking. 

Chain A: 
\begin{verbatim}
./reduce.py --at 1F34_A.pdb --cg 1F34_A.cg.pdb --red at2cg.prot.dat --ff ff_param.dat 
--conv type_conversion.dat 
\end{verbatim}
{\tt 1F34\_A.cg.pdb} contains 684 beads.\\

Chain B:
\begin{verbatim}
./reduce.py --at 1F34_B.pdb --cg 1F34_B.cg.pdb --red at2cg.prot.dat --ff ff_param.dat 
--conv type_conversion.dat 
\end{verbatim}
{\tt 1F34\_B.cg.pdb} 318 beads. \\

The {\tt reduce.py} script required the following input files:

\begin{itemize}
\item an all-atom protein pdb file (option {\tt --at}, for instance {\tt 1F34\_A.pdb})
\item a reduced output pdb file (option {\tt --cg}, for instance {\tt 1F34\_A.cg.pdb})
\item a bead topololgie file (option {\tt --ff}, for instance {\tt at2cg.prot.dat})
\item a forcefield parameter file (option {\tt --ff}, for instance {\tt ff\_param.dat})
\item optionnaly, a filetype conversion file (option {\tt --conv}, for instance {\tt type\_conversion.dat})
\end{itemize}

Visualise both coarse grain partners to check the reduction worked properly.



\subsubsection{Initial ligand positions}

The systematic docking simulation uses starting points placed around the
receptor.  The Python script {\tt translate.py} employs a slightly modified
Shrake and Rupley \cite{shrake1973} method to define starting positions
from receptor surface.  The surface generation fonctions are implemented in
the PTools library. The script first reads the coarse grain (reduced)
receptor and ligand files, then generates a grid of points at the certain
distance from the receptor and outputs the grid with a given density.

Remarks:
\begin{itemize}
\item {\tt translate.py} reads the {\tt aminon.par} parameter file. This file must be
in the directory of the receptor and ligand files.
\item the density option ({\tt -d}) controls the minimum distance between starting
points (in \AA$^2$). The default value is 100.0 (\AA$^2$). 
\item the {\tt -h} options, shows the help message and exit.
\end{itemize}

In the case of the 1F34 complex, the chain A (biggest) is defined as the receptor
and the chain B (smaller) is defined as the ligand.
\begin{verbatim}
./translate.py 1F34_A.cg.pdb 1F34_B.cg.pdb > translation.dat
\end{verbatim}

The visualisation of the starting points may be viewed with any visualisation software by renaming {\tt translation.dat} in {\tt translation.pdb}.

For each position in translation (each line of the file {\tt translation.dat}), 
there are 258 associated rotations defined in the file {\tt rotation.dat}. 

\subsubsection{ATTRACT parameters}

ATTRACT parameters are found in the file {\tt attract.inp}. A typical configuration file is:
\begin{verbatim}
 6 0 0
-34.32940 38.75490 -3.66956 0.00050
 30  2 1 1 0 0 0 0 1 2000.00
 30  2 1 1 0 0 0 0 1 1000.00
 50  2 1 1 0 0 0 0 0  500.00
 50  2 1 1 0 0 0 0 0   50.00
 100 2 1 1 0 0 0 0 0   50.00
 500 2 1 1 0 0 0 0 0   50.00
\end{verbatim}

The first line indicates the number of minimisations performed by ATTRACT for each starting position. 
The last six lines are the caracteristics of each minimisation. The first column is the number of steps before the minimisation stops. 
The last column is the square of the cutoff distance for the calculation of the interaction energy between both partners.

\paragraph{Remarques:} Columns with zeros or ones should not be modfied, as well as the second line.

\subsubsection{Initial docking simulation}

A docking simulation with ATTRACT requires:
\begin{itemize}
\item a receptor (fixed partner) file ({\tt 1F34\_A.cg.pdb})
\item a ligand (mobile partner) file ({\tt 1F34\_B.cg.pdb})
\item the ATTRACT program ({\tt  Attract.py})
\item the coarse grain parameters ({\tt aminon.par})
\item rotations performed for each translational starting point ({\tt rotation.dat})
\item translational starting points ({\tt translation.dat})
\item docking parameters ({\tt attract.inp})
\end{itemize}

ATTRACT can be used with different options:
\begin{itemize}
\item -s, performs one single serie of minimisations with the ligand in its initial position.
\item -ref=, provides a ligand pdb file as a reference. After each docking, the RMSD is calculated between this reference structure and the simulated ligand.
\end{itemize}

A single ATTRACT simulation may thus be obtained by:
\begin{verbatim}
Attract.py 1F34_A.cg.pdb 1F34_B.cg.pdb 
-s -ref=1F34_B.cg.pdb > 1F34_single.att
\end{verbatim}

The first pdb file provided must be the receptor file (and the second the ligand).

For a full systematic docking in the translational and rotational space:
\begin{verbatim}
Attract.py 1F34_A.cg.pdb AF34_B.cg.pdb 
-ref=1F34_B.cg.pdb > 1F34_dock.att
\end{verbatim}

The output file {\tt 1F34\_dock.att} contains all informations on the docking simulation.

\subsubsection{Docking output analysis}

Best geometries found during the docking simulation can be listed with
\begin{verbatim}
cat 1F34_dock.att | egrep -e "^==" | sort -n -k4 | head
\end{verbatim}

Any simulated ligand structure can be extracted with the script {\tt Extract.py}:

\begin{verbatim}
Extract.py 1F34_dock.att 1F34_B.cg.pdb 101 153 > 1F34_dock.pdb
\end{verbatim}

with the parameters:
\begin{itemize}
\item an ouput of the docking simulation ({\tt 1F34\_dock.att})
\item the inital ligand file ({\tt 1F34\_B.cg.z.pdb})
\item a translation index ({\tt 101})
\item a rotation index ({\tt 153})
\item an output ligand structure file ({\tt 1F34\_dock.pdb})
\end{itemize}



%%%%%%%%%%%%%%%%%%%%%%%%%%%%%%
\section{Misc. tips and tricks}


\subsection{Troubleshooting}

\subsubsection{Bus error}
On Mac OS, the command "import ptools" can lead to a "bus error" error message. This happens with MacPorts or 
fink versions of python. Solution: use the python provided with the system instead (/usr/bin/python)



\begin{thebibliography}{99}

\bibitem{shrake1973} A. Shrake, and J.A. Rupley, 
{\it Environment and exposure to solvent of protein atoms. Lysozyme and
isulin}, 
Journal of Molecular Biology, {\bf 79}:351-364, 1973.

\end{thebibliography}
\end{document}


		
